\documentclass[12pt]{article}
\pagestyle{empty}
\usepackage{amsmath, amssymb, amsthm}
\usepackage{latexsym, epsfig, ulem, cancel, multicol, hyperref}
\usepackage{graphicx, tikz, subfigure,pgfplots}
\usepackage[margin=1in]{geometry}
\setlength{\parindent}{0pt}
\usepackage{multirow}
\usepackage{mathtools}
\usepackage{verbatim}
\usepackage{tikz}
\usepackage{pgfplots}
\setlength{\parskip}{1ex}

\newcommand{\T}[0]{\top}
\newcommand{\F}[0]{\bot}
\newcommand{\liminfty}[1]{\lim_{#1 \to \infty}}
\newcommand{\limzero}[1]{\lim_{#1 \to 0}}
\newcommand{\limto}[1]{\lim_{#1}}
\newcommand{\Z}{\mathbb{Z}}
\newcommand{\R}{\mathbb{R}}
\newcommand{\C}{\mathbb{C}}
\newcommand{\Q}{\mathbb{Q}}
\newcommand{\odd}[0]{\mathbb{Z} - 2\mathbb{Z}}
\newcommand{\lineint}[1]{\int_{#1}}
\newcommand{\pypx}[2]{\frac{\partial #1}{\partial #2}}
\newcommand{\divg}{\nabla \cdot}
\newcommand{\curl}{\nabla \times}
\newcommand{\dydx}[2]{\frac{d #1}{d #2}}
\newcommand{\sqbkt}[1]{\left[ #1 \right]}
\newcommand{\paren}[1]{\left( #1 \right)}
\newcommand{\tribkt}[1]{\left< #1 \right>}
\newcommand{\abso}[1]{\left|#1 \right|}
\newcommand{\zero}{\{0\}}
\newcommand{\then}{\rightarrow}
\newcommand{\nonneg}{\Z^+ \cup \{0\}}
\DeclarePairedDelimiter\ceil{\lceil}{\rceil}
\DeclarePairedDelimiter\floor{\lfloor}{\rfloor}
\newcommand{\union}[2]{\bigcup_{#1}^{#2}}
\newcommand{\inter}[2]{\bigcap_{#1}^{#2}}
\newcommand{\openclose}[1]{\left( #1 \right]}
\newcommand{\closeopen}[1]{\left[ #1 \right)}
\newcommand{\compo}[2]{#1 e^{i #2}}
\newcommand{\laplase}{\bigtriangleup}
\newcommand{\bra}[1]{\left< #1 \right|}
\newcommand{\ket}[1]{\left| #1 \right>}
\newcommand{\braket}[2]{\left< #1 \mid #2 \right>}
\newcommand{\ketbra}[2]{\left| #1 \right> \left< #2 \right|}
\newcommand{\ketpsit}{\ket{\psi(t)}}
\newcommand{\ketphit}{\ket{\phi(t)}}
\newcommand{\ham}{\mathbf{H}}
\newcommand{\unx}{\hat{\mathbf{x}}}
\newcommand{\uny}{\hat{\mathbf{y}}}
\newcommand{\unz}{\hat{\mathbf{z}}}
\newcommand{\uni}{\hat{\mathbf{i}}}
\newcommand{\unj}{\hat{\mathbf{j}}}
\newcommand{\unk}{\hat{\mathbf{k}}}
\newcommand{\uns}{\hat{\mathbf{s}}}
\newcommand{\unr}{\hat{\mathbf{r}}}
\newcommand{\untheta}{\hat{\boldsymbol\theta}}
\newcommand{\unphi}{\hat{\boldsymbol\phi}}
\newcommand{\unrho}{\hat{\boldsymbol\rho}}


\newcommand{\wsnumber}{1}
\newcommand{\wstopic}{Vectors}
\pgfplotsset{
    every linear axis/.append style={
       axis x line=center,
       axis y line=center,
       xlabel={$x$},
       ylabel={$y$}
    },
    every axis plot/.append style={thick,mark=none}
}
\tikzset{
    point/.style={circle,draw,fill,minimum width=0.3ex,inner sep=0pt,outer sep=0pt},
    every label/.append style={black}
}


\usepackage[margin=1in]{geometry}
\usepackage{amsmath, amssymb, amsthm, graphicx, hyperref}
\usepackage{enumerate}
\usepackage{fancyhdr}
\usepackage{multirow, multicol}
\usepackage{tikz}
\pagestyle{fancy}
\fancyhead[RO]{Dennis Li}
\fancyhead[LO]{Analytical Mechanics }
\usepackage{comment}
\newif\ifshow
\showfalse

\ifshow
  \newenvironment{solution}{\textbf{Solution.}}{}
\else
  \excludecomment{solution}
\fi

\renewcommand{\thefootnote}{\fnsymbol{footnote}}
\usepackage{comment}


\newtheorem*{remark}{Remark}


\begin{document}
\begin{center}
\ifshow
  \textbf{\Large Problem Set 01 Solution}\\
\else
  \textbf{\Large Problem Set 01}\\
\fi
Instructor \\ Prof. Gabe\\
\end{center}

\hrule

\vspace{0.2cm}

\section{Which statement is frame dependent?}

\begin{enumerate}
    \item A is at rest.

    This is true in $A's$ own reference frame, or a reference frame in which $A$ is at rest. This is not true for all frame, and not true for some inertial frame that is moving at constant velocity with respect to $A$.

    \item $A$ is currently 5 meters away from $B$

    This is true for all frame, and is true for all inertial frame since we are discussing non-relativistic. 

    \item $A$ is moving at 6 $m/s$.

    This is not true in all frame, and not true in all reference fame. Example being the frame $A$ where $A$ is at rest. 

    \item $A$ is moving 6 $m/s$ faster than $B$.

    This is not true for all frame, since in a frame relatively at rest with respect to $A$, $B$ is the only particle moving and $A$ is at rest. And it is not true for all inertial frame since a frame defined just now was inertial. 

    \item $A$ is at the same place as it was 5$s$ ago. 

    This is not true for frames that are moving relative to $A$. It cannot be true for all frame and all inertial frames. 

    \item $A$ is just as far from $B$ as it was 5$s$ ago.

    This is true for all frames and all inertial frames.

    \item $A$'s position relative to $B$ is not changing. 

    This is true for all frames and for all inertial frames. 

    \item $A$’s position relative to $B$ is the same as it was 5 seconds ago.

    This is true for all frames and all inertial frames. 

    \item $A$’s direction of motion is perpendicular to $B$’s.

    This is not true if we define a frame $S$ where $A$ is at rest, then it is not moving and we cannot define orthogonality in this case. So it is not true for all frame and not true for all inertial frame. 

    \item $A$’s direction of motion relative to $C$ is perpendicular to B’s direction of motion relative to $C$.

    \item A is always 3 m away from B.
    
    This is always true for all frames and all inertial frames. 

    \item A’s speed relative to B is 3 m/s.
    
    This is always true for inertial frames but not true for all frames, for example in a spinning reference frame.

    \item A is accelerating.
    
    This is not true for all frames but true for inertial frames.

    \item A is speeding up.
    
    This is not true for all frames and not true for inertial frames.

    \item A’s velocity relative to B is constant.
    
    This is not true for all frames and not true for inertial frames.

    \item A’s velocity relative to B is changing.
    
    This is not true for all frames and not true for inertial frames.

    \item A, B, and C are currently at the vertices of an equilateral triangle.
    
    This is not true for all frames and not true for inertial frames.


    \item A, B, and C are always at the vertices of an equilateral triangle.
    
    This is not true for all frames and not true for inertial frames.

    \item The velocity of A with respect to B is constant.

    This is not true for all frames and not true for inertial frames.

    \item Each of the velocities of A, B, and C makes a 60° angle with the other two.

    This is not true for all frames and not true for inertial frames.

\end{enumerate}

\section{1 falling monkey, 1 projectile}

You may have seen a demo along the following lines in your intro course, or had a textbook problem similar to it…

A monkey is initially hanging from a tree branch a height \(h\) above flat ground. A cannon sits on the ground some distance away from the tree. The cannon fires a projectile at the monkey with initial speed \(v_0\), and the same instant the projectile is launched, the monkey lets go of the tree branch.

\textbf{(a)} Suppose the line-of-sight angle from the ground to the monkey’s initial position in the tree is \(\theta_0\). In order to hit the monkey, should the angle \(\theta\) relative to the ground at which we fire the projectile be \(\theta > \theta_0\), \(\theta = \theta_0\), \(\theta < \theta_0\), or does the answer depend on the circumstances? How do you know?

The answer to this is $\theta = \theta_0$. This is true because after the bullet leaves the muzzle, it experiences no force except for the gravitation force. This gives the bullet a constant downward acceleration of $g$. When the monkey releases its hand from the tree branch, it also experiences no external force except for the gravitational pull. Therefore the monkey is also accelerating downward with the acceleration $g$. The bullet and the monkey is falling down at the same acceleration, meaning they are both in the same \textit{free fall} frame. This implies that their relative motion with respect to each other will not change. So in the monkey's perspective, it will see the bullet flying towards it until it hits. 

\textbf{(b)} Regardless of the correct answer to part (a), it should be pretty clear that, if the launch speed is too low, the projectile will just dribble out and hit the floor quite near the launcher. In other words, the projectile needs some minimum launch speed \(v_0, \text{min}\) to be able to reach the monkey. Find \(v_0, \text{min}\) in two ways and show that you get the same answer.
\begin{figure}[!h]
    \centering
    \includegraphics[width=0.5\linewidth]{Pictures/PS00/monkey_v2.png}
    \caption{Trajectory of Bullet}
    \label{fig:01-2b}
\end{figure}
\begin{itemize}
    \item \textbf{Way 1:} Impose a condition related to elapsed time (I’m being deliberately vague here because I want you to think on your own what that condition needs to be, in words).
    
    Suppose the Monkey is at height $h$. We have the following equation
    \[
    h = \frac{1}{2}gt^2
    \]
    The time it takes the monkey to fall to the ground is given by
    \[
    t_{max} = \sqrt{\frac{2h}{g}}
    \]
    This means that the bullet have to reach the monkey before $t_{max}$, or before the monkey hits the ground. If the parabolic trajectory the bullet travels takes exactly $t_{max}$ to reach the monkey at the same time it hits the ground. We can define the vertical component of the bullet by
    \[
    v_{b,\perp} = v_b\sin\theta_0
    \]
    The time it take the bullet to complete a up-and-down motion can be written as
    \[
    \frac{t_{max}}{2} = \frac{v_b\sin\theta_0}{g}
    \]
    Substitute our expression for $t_{max}$ to this equation, we can find the speed of the bullet needed to meed this minimum criteria.
    \[
    v_b = \frac{\sqrt{2hg}}{2\sin\theta_0} = \frac{1}{\sin\theta_0}\sqrt{\frac{hg}{2}}
    \]

    
    \item \textbf{Way 2:} Impose a condition related to horizontal displacement of the projectile (again, I’m being vague for the same reason).

    Here, we define the horizontal distance between the monkey and the muzzle to be $d_{b,m}$. The bullet has to travel a parabolic curve whose intersection with the ground, or the 2 roots, has to be the muzzle and the position where the monkey falls down. We can find the relationship between $h$ and $d$ defined by 
    \[
    \tan\theta_0 = \frac{h}{d}
    \]
    since the gunner is directly aiming at the monkey. We see that the bullet have to travel the horizontal displacement $d$ while finishing a complete up-and-down motion, and they should complete these motion at the same time. Therefore, we can set up the following equality
    \[
    \frac{d}{v_b\cos\theta_0} = \frac{2v_b\sin\theta_0}{g}
    \]
    cross multiply, we have
    \[
   2 v_b^2\sin\theta_0\cos\theta_0 = dg
    \]
    Note that $\tan\theta_0 = \frac{h}{d}$, we have $d = \frac{h}{\tan\theta_0}$. Substitute
    \[
    2 v_b^2\sin\theta_0\cos\theta_0 = \frac{hg}{\tan\theta_0}
    \]
    We know that $\tan\theta = \frac{\sin\theta}{\cos\theta}$, we can keep doing substitution
    \[
    2 v_b^2\sin\theta_0\cos\theta_0  = \frac{hg\cos\theta_0}{\sin\theta_0}
    \]
    We can cancel the $\cos\theta_0$ term, and simplify
    \[
    v_b = \sqrt{\frac{hg}{2\sin^2\theta_0}} = \frac{1}{\sin\theta_0}\sqrt{\frac{hg}{2}}
    \]
    This is the same result obtained from the first way. 

    
\end{itemize}

\textbf{(c)} For a different perspective on how to arrive at a correct answer to (a), let’s analyze the situation in a frame whose axes are oriented parallel to the ground’s axes but whose origin is attached to the monkey. In this frame (i.e. relative to the monkey), determine:

\begin{enumerate}[i.]
    \item the acceleration of the projectile,

    The bullet is not accelerating with respect to the monkey.
    \[
    \mathbf{a} = \mathbf{o}
    \]
    \item its initial velocity,

    The initial velocity is the same as it would be observed by the gunner, $\mathbf{v}_0$
    \[
    \mathbf{v_0} = v_b\cos\theta_0 \unx + v_b\sin\theta_0 \uny
    \]
    \item its initial position,

    The initial position is $\mathbf{r}_0$
    \[
    \mathbf{r_0} = -d\unx -h\uny
    \]
    \item its velocity as a function of time

    The velocity is constant through time.
    \[
    \mathbf{v}(t) = \mathbf{v}_0
    \]
    \item its position as a function of time.

    The position as a function of time is linear. And it is 
    \[
    \mathbf{r}(t) = \mathbf{r}_0 + \mathbf{v}_0t
    \]


(\textbf{NOTE} that the answers to (i)–(v) are vectors). Finally:


    \item Demonstrate that the projectile will hit the monkey and compute how long after launch the impact takes place (provided, of course, that we had \(v_0 > v_0, \text{min}\)).
    Since the relationship is linear and the position vector is co-linear with the velocity vector, we can express time as
    \[
    t = \frac{\abso{\mathbf{r}_0}}{\abso{\mathbf{v}_0}}
    \]

    
\end{enumerate}


\section{1 circle, 3 frames}

\textbf{(a)} In an inertial frame \(S\) with a standard orthonormal Cartesian \(\hat{x}, \hat{y}, \hat{z}\) basis, a particle moves clockwise at constant speed \(v\) in a circle of radius \(R\) in the \(x\)-\(y\) plane, centered at \((x = 0, y = 0, z = 0)\). At \(t = 0\), the particle is at \((x = R, y = 0, z = 0)\).

\begin{enumerate}
    \item[(i)] In terms of \(v\) and \(R\), what is the angular speed \(\omega\) of the particle in rad/unit time? Go ahead and use \(\omega\) in all your subsequent answers (it’ll reduce how much you have to write and make expressions look less busy).

    \[
    v = r\omega
    \]

    
    \item[(ii)] Write down the position, velocity, and acceleration of this particle as functions of time (these are \textbf{vectors}—express your answers using the Cartesian basis listed above).

    \[
    \begin{cases}
        \mathbf{r} = R\cos(\omega t) \unx - R\sin(\omega t)\uny\\
        \mathbf{v} = -R\omega\sin(\omega t)\unx  - R\omega\cos(\omega t)\uny\\
        \mathbf{a} = -R\omega^2\cos(\omega t) \unx + R\omega^2\sin(\omega t)\uny
    \end{cases}
    \]


    
\end{enumerate}

\textbf{(b)} Now imagine a frame \(S'\) with axes and basis vectors parallel to the corresponding ones in \(S\) and that moves at constant speed \(v\) in the \(-\hat{x}\) direction along the line \(y = -R\) as viewed in frame \(S\), i.e., the speed of \(S'\) as viewed by \(S\) is the same as the translational speed of the circulating particle as viewed by \(S\).

\begin{enumerate}
    \item[(i)] Write down the position, velocity, and acceleration of the same particle as functions of time as viewed in \(S'\). Since the axes of \(S'\) are aligned with those of \(S\), feel free to share the \(\hat{x}, \hat{y}, \hat{z}\) basis vectors between them.

    \[
    \begin{cases}
        \mathbf{r'}(t) = \paren{R\cos\paren{\omega t} + vt}\unx + \paren{-R\sin\paren{\omega t} + R}\uny\\
        \mathbf{v'}(t) = \paren{-R\omega\sin\paren{\omega t}+ v}\unx  -R\omega\cos\paren{\omega t}\uny\\
        \mathbf{a'}(t)  = -R\omega^2\cos(\omega t) \unx + R\omega^2\sin(\omega t)\uny
    \end{cases}
    \]

    
    
    \item[(ii)] In frame \(S'\), what does the trajectory traced out by the particle look like?

    It will look like a cycloid that curves down, concave down. 



    \item[(iii)] In frame \(S'\), what’s going on with the velocity of the particle at the lowest points (where the trajectory has cusps)?

    At the cusps of the motion, the velocity of the particle is 0. Since the velocity vector will be pointing at the opposite direction to where the frame $S'$ is moving, therefore they sort of cancels. 
\end{enumerate}


复制代码
Go watch the following \textbf{black \& white video} from the 1960s about frames-of-reference (watch the whole thing — it’s worth it). There’s a scene around minute 9:00 where Prof. Hume demonstrates precisely this situation. A small ball is moving in a vertical circle at constant speed (because it’s attached to a spinning disc), and then he lets the whole disc move horizontally on a track at constant speed relative to the ground. A Sharpie-type marker on the ball traces out a curve on a piece of plexiglass (that’s stationary relative to the ground) as it coasts by. See the shape drawn out by the marker? You just determined a parametric equation for that shape. The expressions for \( x(t), y(t) \) you just derived trace out a \textbf{cycloid}. There is a nice description and animation on \href{https://en.wikipedia.org/wiki/Cycloid}{this Wikipedia page}.

But appreciate what you’ve just done. Generating the equation of the cycloid starting from frame \(S'\) isn’t easy. But once you realize that, in a frame that co-moves with the center of the rolling disc, the cycloid is just a particle moving in a circle at constant speed, then life becomes easier. All you need to know is (1) how to write down an equation for the trajectory in that co-moving frame (and circles are fairly easy, so we do know how to do this), and (2) how to \textit{translate} your description in this co-moving frame back into the “ground” frame.

\textbf{This is an extremely powerful and recurring theme in the way physics tries to analyze the world!} Don’t default to analyzing everything from scratch. Instead, look for the familiar in the unfamiliar, and avail yourself of tools like frame-shifting to help make things look familiar.

\textbf{(c)} Let’s see what the circle would look like from yet a third point of view. Now imagine a frame \(S''\) (again with standard axes and basis vectors aligned with those of \(S\)) whose origin, as viewed from frame \(S\), moves along the \(z\)-axis of \(S\) in the \(-\unx\) direction at \(v\) (same speed that \(S\) says the circulating particle has). Same drill:

\begin{enumerate}
    \item[(i)] Write down the position, velocity, and acceleration of the same particle as functions of time as viewed in \(S''\).

    \[
    \begin{cases}
        \mathbf{r''}(t)  = R\cos(\omega t) \unx - R\sin(\omega t)\uny + vt\unz \\
        \mathbf{v''}(t)  = -R\omega\sin(\omega t)\unx  - R\omega\cos(\omega t)\uny + v\unz\\
        \mathbf{a''}(t)  = -R\omega^2\cos(\omega t) \unx + R\omega^2\sin(\omega t)\uny
    \end{cases}
    \]

    \item[(ii)] In frame \(S''\), what does the trajectory traced out by the particle look like? What is this shape called? (Yet again, you have used physics to answer a geometry problem.)

    This thing will trace out a helix in the $+\unz$ axis. 

    
    \item[(iii)] In this frame, how does the speed of the particle vary in time?

    There is an extra segment of $v\unz$ that directs the velocity vector in the $\unz$ direction away from the $xy$ plane. 
    
    \item[(iv)] In this frame, what is the ratio of the magnitude of the acceleration to the magnitude of the velocity?

    Let us use $\varphi$ to define the ratio.
    \[
    \varphi = \frac{R\omega^2}{\sqrt{R^2\omega^2+v^2}}
    \]
    Since $v = R\omega$, we can keep simplifying the expression
    \[
    \varphi = \frac{\omega}{\sqrt{2}}
    \]

    
\end{enumerate}


\section{Some non-circles}

\begin{enumerate}
    \item[T1 1.11] \( \mathbf{r}(t) = b\cos(\omega t)\unx + c\sin(\omega t)\uny \)

    This is an eclipse.
    
    \item[T1 1.12]\( \mathbf{r}(t) = b\cos(\omega t)\unx + c\sin(\omega t)\uny + v_0t\unz \)

    This is an elliptical helix. 
\end{enumerate}



\section{1 puck, 3 frames}

\textbf{1.26**} The hallmark of an inertial reference frame is that any object which is subject to zero net force will travel in a straight line at constant speed. To illustrate this, consider the following: I am standing on a level floor at the origin of an inertial frame \(S\) and kick a frictionless puck due north across the floor.

\textbf{(a)} Write down the \(x\) and \(y\) coordinates of the puck as functions of time as seen from my inertial frame. (Use \(x\) and \(y\) axes pointing east and north respectively.) Now consider two more observers, the first at rest in a frame \(S'\) that travels with constant velocity \(v\) due east relative to \(S\), the second at rest in a frame \(S''\) that travels with constant \textit{acceleration} due east relative to \(S\). (All three frames coincide at the moment when I kick the puck, and \(S''\) is at rest relative to \(S\) at that same moment.)

The coordinate as seen in the frame of $S$ will be 
\[
r_S = (0,v_N t)
\]
Where $v_N$ is the speed where the puck is going to the North.




\textbf{(b)} Find the coordinates \(x', y'\) of the puck and describe the puck’s path as seen from \(S'\).

\[
r_{S'} = (-v_E t,v_N t)
\]

The path of the puck is seen as a straight line moving in the Northwest direction. 

\textbf{(c)} Do the same for \(S''\). Which of the frames is inertial?

The $x$ coordinate will move to the West since $S'$ is moving to the East. And for $S''$
\[
r_{S''} = \paren{-v_Et - \frac{1}{2}at^2, v_N t}
\]


The path for $S''$ will be curved, initially moving to Northwest and slowly leaning more and more towards West direciton. And $S''$ is not inertial since it is accelerating with respect to $S$ and $S'$ that are inertial.



\section{1 puck, 1 turntable, 2 frames}
\begin{enumerate}
    \item[T1 1.27]

    The ball in the rotating table perspective will move in from the edge of the table and go in a curved path and leave on the edge, leaving something like a bite off the apple logo.

    \item[T1 1.46] Consider the experiment of Problem 1.27, in which a frictionless puck is slid straight across a rotating turntable through the center \(O\).

\textbf{(a)} Write down the polar coordinates \(r, \phi\) of the puck as functions of time, as measured in the inertial frame \(S\) of an observer on the ground. (Assume that the puck was launched along the axis \(\phi = 0\) at \(t = 0\).)

Since we are initially on the line $\phi = 0$, or basically the $x$ axis, and the puck goes in a straight line in $S$ perspective, it's polar coordinate can be simply written as
\[
P_{Cart} = (R-v t, 0)
\]
Where $v$ is the speed of the puck.

\textbf{(b)} Now write down the polar coordinates \(r', \phi'\) of the puck as measured by an observer (frame \(S'\)) at rest on the turntable. (Choose these coordinates so that \(\phi\) and \(\phi'\) coincide at \(t = 0\).) Describe and sketch the path seen by this second observer. Is the frame \(S'\) inertial?

\[
P_{polar} = \paren{ R - vt, -\omega t}
\]
Where $v$ is the speed of the puck.

\end{enumerate}

\section{Is this frame inertial}
We do not have enough evidence to show this frame is inertial or not. Since there are tension force from the wire hanging the ball and it is not in a completely force free environment. Therefore we do not have enough evidence, that is, force-less environment. 


\section{1 block, 1 wedge, no friction anywhere}

This is the problem we started in class. A block of mass $m$ is initially at rest on a wedge of mass $M$ (also at rest) that makes an angle $\theta$ with the ground. There is no friction either between the block and the wedge or between the wedge and the floor. The system is released from rest. Determine:
\begin{enumerate}
    \item the magnitudes of the accelerations of the block relative to the floor and of the wedge relative to the floor;

    We start by expressing the gravitational force on the block
    \[
    F_{g,m} = mg
    \]
    A a normal force acts on the block from the wedge, and an equal but opposite reaction from the block acts on the wedge. We can find its horizontal component
    \[
   - F_{m,M,x} = \frac{mg}{\tan\theta}
    \]
    And we can derive the acceleration of the wedge horizontal to the floor in the perspective where the block is stationary. 
    \[
   - a_{w,b} = \frac{mg}{M\tan\theta}
    \]
    WIP
        
    \item the magnitude of the contact force that the wedge applies to the block.
        
\end{enumerate}
Express your answers only in terms of $m$, $M$, $\theta$, and constants of nature.




\section{1 train, 1 person, 1 ball, 2 frames}

\textbf{2-41.} A train moves along the tracks at a constant speed \(u\). A woman on the train throws a ball of mass \(m\) straight ahead with a speed \(v\) with respect to herself.

\textbf{(a)} What is the kinetic energy gain of the ball as measured by a person on the train?
\[
E_{gain} = \frac{1}{2}mv^2
\]

\textbf{(b)} By a person standing by the railroad track?
\[
E_{gain} = \frac{1}{2}m\paren{2uv + v^2}
\]

\textbf{(c)} How much work is done by the woman throwing the ball?
\[
W_{women} = \frac{1}{2}mv^2
\]

\textbf{(d)} By the train?
\[
W_{train} = muv
\]


\end{document}