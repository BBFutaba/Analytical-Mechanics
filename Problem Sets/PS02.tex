\documentclass[12pt]{article}
\pagestyle{empty}
\usepackage{amsmath, amssymb, amsthm}
\usepackage{latexsym, epsfig, ulem, cancel, multicol, hyperref}
\usepackage{graphicx, tikz, subfigure,pgfplots}
\usepackage[margin=1in]{geometry}
\setlength{\parindent}{0pt}
\usepackage{multirow}
\usepackage{mathtools}
\usepackage{verbatim}
\usepackage{tikz}
\usepackage{pgfplots}
\setlength{\parskip}{1ex}

\newcommand{\T}[0]{\top}
\newcommand{\F}[0]{\bot}
\newcommand{\liminfty}[1]{\lim_{#1 \to \infty}}
\newcommand{\limzero}[1]{\lim_{#1 \to 0}}
\newcommand{\limto}[1]{\lim_{#1}}
\newcommand{\Z}{\mathbb{Z}}
\newcommand{\R}{\mathbb{R}}
\newcommand{\C}{\mathbb{C}}
\newcommand{\Q}{\mathbb{Q}}
\newcommand{\odd}[0]{\mathbb{Z} - 2\mathbb{Z}}
\newcommand{\lineint}[1]{\int_{#1}}
\newcommand{\pypx}[2]{\frac{\partial #1}{\partial #2}}
\newcommand{\divg}{\nabla \cdot}
\newcommand{\curl}{\nabla \times}
\newcommand{\dydx}[2]{\frac{d #1}{d #2}}
\newcommand{\sqbkt}[1]{\left[ #1 \right]}
\newcommand{\paren}[1]{\left( #1 \right)}
\newcommand{\tribkt}[1]{\left< #1 \right>}
\newcommand{\abso}[1]{\left|#1 \right|}
\newcommand{\zero}{\{0\}}
\newcommand{\then}{\rightarrow}
\newcommand{\nonneg}{\Z^+ \cup \{0\}}
\DeclarePairedDelimiter\ceil{\lceil}{\rceil}
\DeclarePairedDelimiter\floor{\lfloor}{\rfloor}
\newcommand{\union}[2]{\bigcup_{#1}^{#2}}
\newcommand{\inter}[2]{\bigcap_{#1}^{#2}}
\newcommand{\openclose}[1]{\left( #1 \right]}
\newcommand{\closeopen}[1]{\left[ #1 \right)}
\newcommand{\compo}[2]{#1 e^{i #2}}
\newcommand{\laplase}{\bigtriangleup}
\newcommand{\bra}[1]{\left< #1 \right|}
\newcommand{\ket}[1]{\left| #1 \right>}
\newcommand{\braket}[2]{\left< #1 \mid #2 \right>}
\newcommand{\ketbra}[2]{\left| #1 \right> \left< #2 \right|}
\newcommand{\ketpsit}{\ket{\psi(t)}}
\newcommand{\ketphit}{\ket{\phi(t)}}
\newcommand{\ham}{\mathbf{H}}
\newcommand{\unx}{\hat{\mathbf{x}}}
\newcommand{\uny}{\hat{\mathbf{y}}}
\newcommand{\uns}{\hat{\mathbf{s}}}
\newcommand{\unr}{\hat{\mathbf{r}}}
\newcommand{\untheta}{\hat{\boldsymbol\theta}}
\newcommand{\unphi}{\hat{\boldsymbol\phi}}

\newcommand{\wsnumber}{1}
\newcommand{\wstopic}{Vectors}
\pgfplotsset{
    every linear axis/.append style={
       axis x line=center,
       axis y line=center,
       xlabel={$x$},
       ylabel={$y$}
    },
    every axis plot/.append style={thick,mark=none}
}
\tikzset{
    point/.style={circle,draw,fill,minimum width=0.3ex,inner sep=0pt,outer sep=0pt},
    every label/.append style={black}
}


\usepackage[margin=1in]{geometry}
\usepackage{amsmath, amssymb, amsthm, graphicx, hyperref}
\usepackage{enumerate}
\usepackage{fancyhdr}
\usepackage{multirow, multicol}
\usepackage{tikz}
\pagestyle{fancy}
\fancyhead[RO]{Dennis Li}
\fancyhead[LO]{Analytical Mechanics }
\usepackage{comment}
\newif\ifshow
\showfalse

\ifshow
  \newenvironment{solution}{\textbf{Solution.}}{}
\else
  \excludecomment{solution}
\fi

\renewcommand{\thefootnote}{\fnsymbol{footnote}}
\usepackage{comment}


\newtheorem*{remark}{Remark}


\begin{document}
\begin{center}
\ifshow
  \textbf{\Large Problem Set 02 Solution}\\
\else
  \textbf{\Large Problem Set 02}\\
\fi
Instructor \\ Prof. Gabe\\
\end{center}

\hrule

\vspace{0.2cm}

\section{Vector Kinematics, Encore: Constrained Motion}

\subsection{1 Circle, 1 particle, in a hurry}


\subsection{Osculating Circles}
As discussed in class, we can use kinematics to solve a particular geometry problem, namely to find the “best fit circle” or \textit{osculating circle} to a curve at a given point on that curve. Let’s do this again but a little differently (and more efficiently). We’ll work through a couple of extra cases to illustrate the principle and to give you practice implicitly differentiating equations with respect to time.

\begin{enumerate}
    \item[(a)] Suppose a particle moves at constant speed \( v \) in the plane counterclockwise along the circle \( x^2 + y^2 = R^2 \). When the particle is at a position 120\(^\circ\) away from the positive \(x\)-axis along the arc of the circle, what is its velocity\(^1\)? What is its acceleration\(^2\)? What is the radius of the osculating circle to the curve at this point? (Spoiler alert — that radius should come out to \( R \), since the best-fit circle to a circular curve is the circle itself.)

    You could do this problem just with what you know kinematically about how the velocity and acceleration are related for uniform circular motion, but instead, try to do it by differentiating the equation of the circle with respect to \( t \) a couple of times, interpreting what you get, and applying other constraints (e.g., that \( \dot{x}(t)^2 + \dot{y}(t)^2 = v^2 \) is constant in this problem). In general, you may not have physical intuition about how these vectors’ magnitudes are related on arbitrary curves under arbitrary conditions, so it’s good to practice a more generalizable method in a situation where you can obtain the answer through other means, just so you can gut-check your usage of this more general (and potentially more unfamiliar) method.

    The velocity can simply be written as, with $v_0$ being the constant speed.
    \[
    \mathbf{v} = -R\omega\sin\omega t \unx + R\omega\cos\omega t\uny
    \]
    Where $v_0 = R\omega$. And the acceleration is
    \[
    \mathbf{v} = -R\omega^2\cos\omega t \unx - R\omega^2\sin\omega t\uny
    \]
    And the radius of the osculating circle is $R$. We can also obtain this by doing implicit differentiation
    \[
    2x\dot{x} + 2y\dot{y} = 0
    \]
    We also know that the magnitude of the velocity is constant, so
    \[
    \dot{x}^2+\dot{y}^2 = v_0^2
    \]
    This means
    \[
    x\dot{x}+y\dot{y} = 0
    \]
    We can solve this to obtain the same result.

    \item[(b)] Now take a particle moving at constant speed \( v_0 \) along the curve \( y = \cos x + 3 \) such that its \(x\)-coordinate increases continuously with time and such that the particle is at \( (x = 0, y = 4) \) when \( t = 0 \) (at negative times, the particle was passing through negative-\(x\) values).

    \begin{enumerate}
        \item[(i)] Determine the velocity \( \vec{v} \) and acceleration \( \vec{a} \) of the particle and the radius \( r_{\text{osc}} \) of the osculating circle to the curve when the particle is at the points with the following \(x\)-coordinates:
        \begin{itemize}
            \item \( x = 0 \)
            \item \( x = \pi/2 \)
            \item \( x = \pi \)
        \end{itemize}

        \item[(ii)] If the frame in which the motion is observed to have this constant speed \( v_0 \) along this curve is inertial, then is a frame that moves along with the particle inertial? Justify your answer.
    \end{enumerate}
\end{enumerate}




\end{document}